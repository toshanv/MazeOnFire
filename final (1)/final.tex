\documentclass[11pt]{article}

    \usepackage[breakable]{tcolorbox}
    \usepackage{parskip} % Stop auto-indenting (to mimic markdown behaviour)
    
    \usepackage{iftex}
    \ifPDFTeX
    	\usepackage[T1]{fontenc}
    	\usepackage{mathpazo}
    \else
    	\usepackage{fontspec}
    \fi

    % Basic figure setup, for now with no caption control since it's done
    % automatically by Pandoc (which extracts ![](path) syntax from Markdown).
    \usepackage{graphicx}
    % Maintain compatibility with old templates. Remove in nbconvert 6.0
    \let\Oldincludegraphics\includegraphics
    % Ensure that by default, figures have no caption (until we provide a
    % proper Figure object with a Caption API and a way to capture that
    % in the conversion process - todo).
    \usepackage{caption}
    \DeclareCaptionFormat{nocaption}{}
    \captionsetup{format=nocaption,aboveskip=0pt,belowskip=0pt}

    \usepackage{float}
    \floatplacement{figure}{H} % forces figures to be placed at the correct location
    \usepackage{xcolor} % Allow colors to be defined
    \usepackage{enumerate} % Needed for markdown enumerations to work
    \usepackage{geometry} % Used to adjust the document margins
    \usepackage{amsmath} % Equations
    \usepackage{amssymb} % Equations
    \usepackage{textcomp} % defines textquotesingle
    % Hack from http://tex.stackexchange.com/a/47451/13684:
    \AtBeginDocument{%
        \def\PYZsq{\textquotesingle}% Upright quotes in Pygmentized code
    }
    \usepackage{upquote} % Upright quotes for verbatim code
    \usepackage{eurosym} % defines \euro
    \usepackage[mathletters]{ucs} % Extended unicode (utf-8) support
    \usepackage{fancyvrb} % verbatim replacement that allows latex
    \usepackage{grffile} % extends the file name processing of package graphics 
                         % to support a larger range
    \makeatletter % fix for old versions of grffile with XeLaTeX
    \@ifpackagelater{grffile}{2019/11/01}
    {
      % Do nothing on new versions
    }
    {
      \def\Gread@@xetex#1{%
        \IfFileExists{"\Gin@base".bb}%
        {\Gread@eps{\Gin@base.bb}}%
        {\Gread@@xetex@aux#1}%
      }
    }
    \makeatother
    \usepackage[Export]{adjustbox} % Used to constrain images to a maximum size
    \adjustboxset{max size={0.9\linewidth}{0.9\paperheight}}

    % The hyperref package gives us a pdf with properly built
    % internal navigation ('pdf bookmarks' for the table of contents,
    % internal cross-reference links, web links for URLs, etc.)
    \usepackage{hyperref}
    % The default LaTeX title has an obnoxious amount of whitespace. By default,
    % titling removes some of it. It also provides customization options.
    \usepackage{titling}
    \usepackage{longtable} % longtable support required by pandoc >1.10
    \usepackage{booktabs}  % table support for pandoc > 1.12.2
    \usepackage[inline]{enumitem} % IRkernel/repr support (it uses the enumerate* environment)
    \usepackage[normalem]{ulem} % ulem is needed to support strikethroughs (\sout)
                                % normalem makes italics be italics, not underlines
    \usepackage{mathrsfs}
    

    
    % Colors for the hyperref package
    \definecolor{urlcolor}{rgb}{0,.145,.698}
    \definecolor{linkcolor}{rgb}{.71,0.21,0.01}
    \definecolor{citecolor}{rgb}{.12,.54,.11}

    % ANSI colors
    \definecolor{ansi-black}{HTML}{3E424D}
    \definecolor{ansi-black-intense}{HTML}{282C36}
    \definecolor{ansi-red}{HTML}{E75C58}
    \definecolor{ansi-red-intense}{HTML}{B22B31}
    \definecolor{ansi-green}{HTML}{00A250}
    \definecolor{ansi-green-intense}{HTML}{007427}
    \definecolor{ansi-yellow}{HTML}{DDB62B}
    \definecolor{ansi-yellow-intense}{HTML}{B27D12}
    \definecolor{ansi-blue}{HTML}{208FFB}
    \definecolor{ansi-blue-intense}{HTML}{0065CA}
    \definecolor{ansi-magenta}{HTML}{D160C4}
    \definecolor{ansi-magenta-intense}{HTML}{A03196}
    \definecolor{ansi-cyan}{HTML}{60C6C8}
    \definecolor{ansi-cyan-intense}{HTML}{258F8F}
    \definecolor{ansi-white}{HTML}{C5C1B4}
    \definecolor{ansi-white-intense}{HTML}{A1A6B2}
    \definecolor{ansi-default-inverse-fg}{HTML}{FFFFFF}
    \definecolor{ansi-default-inverse-bg}{HTML}{000000}

    % common color for the border for error outputs.
    \definecolor{outerrorbackground}{HTML}{FFDFDF}

    % commands and environments needed by pandoc snippets
    % extracted from the output of `pandoc -s`
    \providecommand{\tightlist}{%
      \setlength{\itemsep}{0pt}\setlength{\parskip}{0pt}}
    \DefineVerbatimEnvironment{Highlighting}{Verbatim}{commandchars=\\\{\}}
    % Add ',fontsize=\small' for more characters per line
    \newenvironment{Shaded}{}{}
    \newcommand{\KeywordTok}[1]{\textcolor[rgb]{0.00,0.44,0.13}{\textbf{{#1}}}}
    \newcommand{\DataTypeTok}[1]{\textcolor[rgb]{0.56,0.13,0.00}{{#1}}}
    \newcommand{\DecValTok}[1]{\textcolor[rgb]{0.25,0.63,0.44}{{#1}}}
    \newcommand{\BaseNTok}[1]{\textcolor[rgb]{0.25,0.63,0.44}{{#1}}}
    \newcommand{\FloatTok}[1]{\textcolor[rgb]{0.25,0.63,0.44}{{#1}}}
    \newcommand{\CharTok}[1]{\textcolor[rgb]{0.25,0.44,0.63}{{#1}}}
    \newcommand{\StringTok}[1]{\textcolor[rgb]{0.25,0.44,0.63}{{#1}}}
    \newcommand{\CommentTok}[1]{\textcolor[rgb]{0.38,0.63,0.69}{\textit{{#1}}}}
    \newcommand{\OtherTok}[1]{\textcolor[rgb]{0.00,0.44,0.13}{{#1}}}
    \newcommand{\AlertTok}[1]{\textcolor[rgb]{1.00,0.00,0.00}{\textbf{{#1}}}}
    \newcommand{\FunctionTok}[1]{\textcolor[rgb]{0.02,0.16,0.49}{{#1}}}
    \newcommand{\RegionMarkerTok}[1]{{#1}}
    \newcommand{\ErrorTok}[1]{\textcolor[rgb]{1.00,0.00,0.00}{\textbf{{#1}}}}
    \newcommand{\NormalTok}[1]{{#1}}
    
    % Additional commands for more recent versions of Pandoc
    \newcommand{\ConstantTok}[1]{\textcolor[rgb]{0.53,0.00,0.00}{{#1}}}
    \newcommand{\SpecialCharTok}[1]{\textcolor[rgb]{0.25,0.44,0.63}{{#1}}}
    \newcommand{\VerbatimStringTok}[1]{\textcolor[rgb]{0.25,0.44,0.63}{{#1}}}
    \newcommand{\SpecialStringTok}[1]{\textcolor[rgb]{0.73,0.40,0.53}{{#1}}}
    \newcommand{\ImportTok}[1]{{#1}}
    \newcommand{\DocumentationTok}[1]{\textcolor[rgb]{0.73,0.13,0.13}{\textit{{#1}}}}
    \newcommand{\AnnotationTok}[1]{\textcolor[rgb]{0.38,0.63,0.69}{\textbf{\textit{{#1}}}}}
    \newcommand{\CommentVarTok}[1]{\textcolor[rgb]{0.38,0.63,0.69}{\textbf{\textit{{#1}}}}}
    \newcommand{\VariableTok}[1]{\textcolor[rgb]{0.10,0.09,0.49}{{#1}}}
    \newcommand{\ControlFlowTok}[1]{\textcolor[rgb]{0.00,0.44,0.13}{\textbf{{#1}}}}
    \newcommand{\OperatorTok}[1]{\textcolor[rgb]{0.40,0.40,0.40}{{#1}}}
    \newcommand{\BuiltInTok}[1]{{#1}}
    \newcommand{\ExtensionTok}[1]{{#1}}
    \newcommand{\PreprocessorTok}[1]{\textcolor[rgb]{0.74,0.48,0.00}{{#1}}}
    \newcommand{\AttributeTok}[1]{\textcolor[rgb]{0.49,0.56,0.16}{{#1}}}
    \newcommand{\InformationTok}[1]{\textcolor[rgb]{0.38,0.63,0.69}{\textbf{\textit{{#1}}}}}
    \newcommand{\WarningTok}[1]{\textcolor[rgb]{0.38,0.63,0.69}{\textbf{\textit{{#1}}}}}
    
    
    % Define a nice break command that doesn't care if a line doesn't already
    % exist.
    \def\br{\hspace*{\fill} \\* }
    % Math Jax compatibility definitions
    \def\gt{>}
    \def\lt{<}
    \let\Oldtex\TeX
    \let\Oldlatex\LaTeX
    \renewcommand{\TeX}{\textrm{\Oldtex}}
    \renewcommand{\LaTeX}{\textrm{\Oldlatex}}
    % Document parameters
    % Document title
    \title{final}
    
    
    
    
    
% Pygments definitions
\makeatletter
\def\PY@reset{\let\PY@it=\relax \let\PY@bf=\relax%
    \let\PY@ul=\relax \let\PY@tc=\relax%
    \let\PY@bc=\relax \let\PY@ff=\relax}
\def\PY@tok#1{\csname PY@tok@#1\endcsname}
\def\PY@toks#1+{\ifx\relax#1\empty\else%
    \PY@tok{#1}\expandafter\PY@toks\fi}
\def\PY@do#1{\PY@bc{\PY@tc{\PY@ul{%
    \PY@it{\PY@bf{\PY@ff{#1}}}}}}}
\def\PY#1#2{\PY@reset\PY@toks#1+\relax+\PY@do{#2}}

\expandafter\def\csname PY@tok@w\endcsname{\def\PY@tc##1{\textcolor[rgb]{0.73,0.73,0.73}{##1}}}
\expandafter\def\csname PY@tok@c\endcsname{\let\PY@it=\textit\def\PY@tc##1{\textcolor[rgb]{0.25,0.50,0.50}{##1}}}
\expandafter\def\csname PY@tok@cp\endcsname{\def\PY@tc##1{\textcolor[rgb]{0.74,0.48,0.00}{##1}}}
\expandafter\def\csname PY@tok@k\endcsname{\let\PY@bf=\textbf\def\PY@tc##1{\textcolor[rgb]{0.00,0.50,0.00}{##1}}}
\expandafter\def\csname PY@tok@kp\endcsname{\def\PY@tc##1{\textcolor[rgb]{0.00,0.50,0.00}{##1}}}
\expandafter\def\csname PY@tok@kt\endcsname{\def\PY@tc##1{\textcolor[rgb]{0.69,0.00,0.25}{##1}}}
\expandafter\def\csname PY@tok@o\endcsname{\def\PY@tc##1{\textcolor[rgb]{0.40,0.40,0.40}{##1}}}
\expandafter\def\csname PY@tok@ow\endcsname{\let\PY@bf=\textbf\def\PY@tc##1{\textcolor[rgb]{0.67,0.13,1.00}{##1}}}
\expandafter\def\csname PY@tok@nb\endcsname{\def\PY@tc##1{\textcolor[rgb]{0.00,0.50,0.00}{##1}}}
\expandafter\def\csname PY@tok@nf\endcsname{\def\PY@tc##1{\textcolor[rgb]{0.00,0.00,1.00}{##1}}}
\expandafter\def\csname PY@tok@nc\endcsname{\let\PY@bf=\textbf\def\PY@tc##1{\textcolor[rgb]{0.00,0.00,1.00}{##1}}}
\expandafter\def\csname PY@tok@nn\endcsname{\let\PY@bf=\textbf\def\PY@tc##1{\textcolor[rgb]{0.00,0.00,1.00}{##1}}}
\expandafter\def\csname PY@tok@ne\endcsname{\let\PY@bf=\textbf\def\PY@tc##1{\textcolor[rgb]{0.82,0.25,0.23}{##1}}}
\expandafter\def\csname PY@tok@nv\endcsname{\def\PY@tc##1{\textcolor[rgb]{0.10,0.09,0.49}{##1}}}
\expandafter\def\csname PY@tok@no\endcsname{\def\PY@tc##1{\textcolor[rgb]{0.53,0.00,0.00}{##1}}}
\expandafter\def\csname PY@tok@nl\endcsname{\def\PY@tc##1{\textcolor[rgb]{0.63,0.63,0.00}{##1}}}
\expandafter\def\csname PY@tok@ni\endcsname{\let\PY@bf=\textbf\def\PY@tc##1{\textcolor[rgb]{0.60,0.60,0.60}{##1}}}
\expandafter\def\csname PY@tok@na\endcsname{\def\PY@tc##1{\textcolor[rgb]{0.49,0.56,0.16}{##1}}}
\expandafter\def\csname PY@tok@nt\endcsname{\let\PY@bf=\textbf\def\PY@tc##1{\textcolor[rgb]{0.00,0.50,0.00}{##1}}}
\expandafter\def\csname PY@tok@nd\endcsname{\def\PY@tc##1{\textcolor[rgb]{0.67,0.13,1.00}{##1}}}
\expandafter\def\csname PY@tok@s\endcsname{\def\PY@tc##1{\textcolor[rgb]{0.73,0.13,0.13}{##1}}}
\expandafter\def\csname PY@tok@sd\endcsname{\let\PY@it=\textit\def\PY@tc##1{\textcolor[rgb]{0.73,0.13,0.13}{##1}}}
\expandafter\def\csname PY@tok@si\endcsname{\let\PY@bf=\textbf\def\PY@tc##1{\textcolor[rgb]{0.73,0.40,0.53}{##1}}}
\expandafter\def\csname PY@tok@se\endcsname{\let\PY@bf=\textbf\def\PY@tc##1{\textcolor[rgb]{0.73,0.40,0.13}{##1}}}
\expandafter\def\csname PY@tok@sr\endcsname{\def\PY@tc##1{\textcolor[rgb]{0.73,0.40,0.53}{##1}}}
\expandafter\def\csname PY@tok@ss\endcsname{\def\PY@tc##1{\textcolor[rgb]{0.10,0.09,0.49}{##1}}}
\expandafter\def\csname PY@tok@sx\endcsname{\def\PY@tc##1{\textcolor[rgb]{0.00,0.50,0.00}{##1}}}
\expandafter\def\csname PY@tok@m\endcsname{\def\PY@tc##1{\textcolor[rgb]{0.40,0.40,0.40}{##1}}}
\expandafter\def\csname PY@tok@gh\endcsname{\let\PY@bf=\textbf\def\PY@tc##1{\textcolor[rgb]{0.00,0.00,0.50}{##1}}}
\expandafter\def\csname PY@tok@gu\endcsname{\let\PY@bf=\textbf\def\PY@tc##1{\textcolor[rgb]{0.50,0.00,0.50}{##1}}}
\expandafter\def\csname PY@tok@gd\endcsname{\def\PY@tc##1{\textcolor[rgb]{0.63,0.00,0.00}{##1}}}
\expandafter\def\csname PY@tok@gi\endcsname{\def\PY@tc##1{\textcolor[rgb]{0.00,0.63,0.00}{##1}}}
\expandafter\def\csname PY@tok@gr\endcsname{\def\PY@tc##1{\textcolor[rgb]{1.00,0.00,0.00}{##1}}}
\expandafter\def\csname PY@tok@ge\endcsname{\let\PY@it=\textit}
\expandafter\def\csname PY@tok@gs\endcsname{\let\PY@bf=\textbf}
\expandafter\def\csname PY@tok@gp\endcsname{\let\PY@bf=\textbf\def\PY@tc##1{\textcolor[rgb]{0.00,0.00,0.50}{##1}}}
\expandafter\def\csname PY@tok@go\endcsname{\def\PY@tc##1{\textcolor[rgb]{0.53,0.53,0.53}{##1}}}
\expandafter\def\csname PY@tok@gt\endcsname{\def\PY@tc##1{\textcolor[rgb]{0.00,0.27,0.87}{##1}}}
\expandafter\def\csname PY@tok@err\endcsname{\def\PY@bc##1{\setlength{\fboxsep}{0pt}\fcolorbox[rgb]{1.00,0.00,0.00}{1,1,1}{\strut ##1}}}
\expandafter\def\csname PY@tok@kc\endcsname{\let\PY@bf=\textbf\def\PY@tc##1{\textcolor[rgb]{0.00,0.50,0.00}{##1}}}
\expandafter\def\csname PY@tok@kd\endcsname{\let\PY@bf=\textbf\def\PY@tc##1{\textcolor[rgb]{0.00,0.50,0.00}{##1}}}
\expandafter\def\csname PY@tok@kn\endcsname{\let\PY@bf=\textbf\def\PY@tc##1{\textcolor[rgb]{0.00,0.50,0.00}{##1}}}
\expandafter\def\csname PY@tok@kr\endcsname{\let\PY@bf=\textbf\def\PY@tc##1{\textcolor[rgb]{0.00,0.50,0.00}{##1}}}
\expandafter\def\csname PY@tok@bp\endcsname{\def\PY@tc##1{\textcolor[rgb]{0.00,0.50,0.00}{##1}}}
\expandafter\def\csname PY@tok@fm\endcsname{\def\PY@tc##1{\textcolor[rgb]{0.00,0.00,1.00}{##1}}}
\expandafter\def\csname PY@tok@vc\endcsname{\def\PY@tc##1{\textcolor[rgb]{0.10,0.09,0.49}{##1}}}
\expandafter\def\csname PY@tok@vg\endcsname{\def\PY@tc##1{\textcolor[rgb]{0.10,0.09,0.49}{##1}}}
\expandafter\def\csname PY@tok@vi\endcsname{\def\PY@tc##1{\textcolor[rgb]{0.10,0.09,0.49}{##1}}}
\expandafter\def\csname PY@tok@vm\endcsname{\def\PY@tc##1{\textcolor[rgb]{0.10,0.09,0.49}{##1}}}
\expandafter\def\csname PY@tok@sa\endcsname{\def\PY@tc##1{\textcolor[rgb]{0.73,0.13,0.13}{##1}}}
\expandafter\def\csname PY@tok@sb\endcsname{\def\PY@tc##1{\textcolor[rgb]{0.73,0.13,0.13}{##1}}}
\expandafter\def\csname PY@tok@sc\endcsname{\def\PY@tc##1{\textcolor[rgb]{0.73,0.13,0.13}{##1}}}
\expandafter\def\csname PY@tok@dl\endcsname{\def\PY@tc##1{\textcolor[rgb]{0.73,0.13,0.13}{##1}}}
\expandafter\def\csname PY@tok@s2\endcsname{\def\PY@tc##1{\textcolor[rgb]{0.73,0.13,0.13}{##1}}}
\expandafter\def\csname PY@tok@sh\endcsname{\def\PY@tc##1{\textcolor[rgb]{0.73,0.13,0.13}{##1}}}
\expandafter\def\csname PY@tok@s1\endcsname{\def\PY@tc##1{\textcolor[rgb]{0.73,0.13,0.13}{##1}}}
\expandafter\def\csname PY@tok@mb\endcsname{\def\PY@tc##1{\textcolor[rgb]{0.40,0.40,0.40}{##1}}}
\expandafter\def\csname PY@tok@mf\endcsname{\def\PY@tc##1{\textcolor[rgb]{0.40,0.40,0.40}{##1}}}
\expandafter\def\csname PY@tok@mh\endcsname{\def\PY@tc##1{\textcolor[rgb]{0.40,0.40,0.40}{##1}}}
\expandafter\def\csname PY@tok@mi\endcsname{\def\PY@tc##1{\textcolor[rgb]{0.40,0.40,0.40}{##1}}}
\expandafter\def\csname PY@tok@il\endcsname{\def\PY@tc##1{\textcolor[rgb]{0.40,0.40,0.40}{##1}}}
\expandafter\def\csname PY@tok@mo\endcsname{\def\PY@tc##1{\textcolor[rgb]{0.40,0.40,0.40}{##1}}}
\expandafter\def\csname PY@tok@ch\endcsname{\let\PY@it=\textit\def\PY@tc##1{\textcolor[rgb]{0.25,0.50,0.50}{##1}}}
\expandafter\def\csname PY@tok@cm\endcsname{\let\PY@it=\textit\def\PY@tc##1{\textcolor[rgb]{0.25,0.50,0.50}{##1}}}
\expandafter\def\csname PY@tok@cpf\endcsname{\let\PY@it=\textit\def\PY@tc##1{\textcolor[rgb]{0.25,0.50,0.50}{##1}}}
\expandafter\def\csname PY@tok@c1\endcsname{\let\PY@it=\textit\def\PY@tc##1{\textcolor[rgb]{0.25,0.50,0.50}{##1}}}
\expandafter\def\csname PY@tok@cs\endcsname{\let\PY@it=\textit\def\PY@tc##1{\textcolor[rgb]{0.25,0.50,0.50}{##1}}}

\def\PYZbs{\char`\\}
\def\PYZus{\char`\_}
\def\PYZob{\char`\{}
\def\PYZcb{\char`\}}
\def\PYZca{\char`\^}
\def\PYZam{\char`\&}
\def\PYZlt{\char`\<}
\def\PYZgt{\char`\>}
\def\PYZsh{\char`\#}
\def\PYZpc{\char`\%}
\def\PYZdl{\char`\$}
\def\PYZhy{\char`\-}
\def\PYZsq{\char`\'}
\def\PYZdq{\char`\"}
\def\PYZti{\char`\~}
% for compatibility with earlier versions
\def\PYZat{@}
\def\PYZlb{[}
\def\PYZrb{]}
\makeatother


    % For linebreaks inside Verbatim environment from package fancyvrb. 
    \makeatletter
        \newbox\Wrappedcontinuationbox 
        \newbox\Wrappedvisiblespacebox 
        \newcommand*\Wrappedvisiblespace {\textcolor{red}{\textvisiblespace}} 
        \newcommand*\Wrappedcontinuationsymbol {\textcolor{red}{\llap{\tiny$\m@th\hookrightarrow$}}} 
        \newcommand*\Wrappedcontinuationindent {3ex } 
        \newcommand*\Wrappedafterbreak {\kern\Wrappedcontinuationindent\copy\Wrappedcontinuationbox} 
        % Take advantage of the already applied Pygments mark-up to insert 
        % potential linebreaks for TeX processing. 
        %        {, <, #, %, $, ' and ": go to next line. 
        %        _, }, ^, &, >, - and ~: stay at end of broken line. 
        % Use of \textquotesingle for straight quote. 
        \newcommand*\Wrappedbreaksatspecials {% 
            \def\PYGZus{\discretionary{\char`\_}{\Wrappedafterbreak}{\char`\_}}% 
            \def\PYGZob{\discretionary{}{\Wrappedafterbreak\char`\{}{\char`\{}}% 
            \def\PYGZcb{\discretionary{\char`\}}{\Wrappedafterbreak}{\char`\}}}% 
            \def\PYGZca{\discretionary{\char`\^}{\Wrappedafterbreak}{\char`\^}}% 
            \def\PYGZam{\discretionary{\char`\&}{\Wrappedafterbreak}{\char`\&}}% 
            \def\PYGZlt{\discretionary{}{\Wrappedafterbreak\char`\<}{\char`\<}}% 
            \def\PYGZgt{\discretionary{\char`\>}{\Wrappedafterbreak}{\char`\>}}% 
            \def\PYGZsh{\discretionary{}{\Wrappedafterbreak\char`\#}{\char`\#}}% 
            \def\PYGZpc{\discretionary{}{\Wrappedafterbreak\char`\%}{\char`\%}}% 
            \def\PYGZdl{\discretionary{}{\Wrappedafterbreak\char`\$}{\char`\$}}% 
            \def\PYGZhy{\discretionary{\char`\-}{\Wrappedafterbreak}{\char`\-}}% 
            \def\PYGZsq{\discretionary{}{\Wrappedafterbreak\textquotesingle}{\textquotesingle}}% 
            \def\PYGZdq{\discretionary{}{\Wrappedafterbreak\char`\"}{\char`\"}}% 
            \def\PYGZti{\discretionary{\char`\~}{\Wrappedafterbreak}{\char`\~}}% 
        } 
        % Some characters . , ; ? ! / are not pygmentized. 
        % This macro makes them "active" and they will insert potential linebreaks 
        \newcommand*\Wrappedbreaksatpunct {% 
            \lccode`\~`\.\lowercase{\def~}{\discretionary{\hbox{\char`\.}}{\Wrappedafterbreak}{\hbox{\char`\.}}}% 
            \lccode`\~`\,\lowercase{\def~}{\discretionary{\hbox{\char`\,}}{\Wrappedafterbreak}{\hbox{\char`\,}}}% 
            \lccode`\~`\;\lowercase{\def~}{\discretionary{\hbox{\char`\;}}{\Wrappedafterbreak}{\hbox{\char`\;}}}% 
            \lccode`\~`\:\lowercase{\def~}{\discretionary{\hbox{\char`\:}}{\Wrappedafterbreak}{\hbox{\char`\:}}}% 
            \lccode`\~`\?\lowercase{\def~}{\discretionary{\hbox{\char`\?}}{\Wrappedafterbreak}{\hbox{\char`\?}}}% 
            \lccode`\~`\!\lowercase{\def~}{\discretionary{\hbox{\char`\!}}{\Wrappedafterbreak}{\hbox{\char`\!}}}% 
            \lccode`\~`\/\lowercase{\def~}{\discretionary{\hbox{\char`\/}}{\Wrappedafterbreak}{\hbox{\char`\/}}}% 
            \catcode`\.\active
            \catcode`\,\active 
            \catcode`\;\active
            \catcode`\:\active
            \catcode`\?\active
            \catcode`\!\active
            \catcode`\/\active 
            \lccode`\~`\~ 	
        }
    \makeatother

    \let\OriginalVerbatim=\Verbatim
    \makeatletter
    \renewcommand{\Verbatim}[1][1]{%
        %\parskip\z@skip
        \sbox\Wrappedcontinuationbox {\Wrappedcontinuationsymbol}%
        \sbox\Wrappedvisiblespacebox {\FV@SetupFont\Wrappedvisiblespace}%
        \def\FancyVerbFormatLine ##1{\hsize\linewidth
            \vtop{\raggedright\hyphenpenalty\z@\exhyphenpenalty\z@
                \doublehyphendemerits\z@\finalhyphendemerits\z@
                \strut ##1\strut}%
        }%
        % If the linebreak is at a space, the latter will be displayed as visible
        % space at end of first line, and a continuation symbol starts next line.
        % Stretch/shrink are however usually zero for typewriter font.
        \def\FV@Space {%
            \nobreak\hskip\z@ plus\fontdimen3\font minus\fontdimen4\font
            \discretionary{\copy\Wrappedvisiblespacebox}{\Wrappedafterbreak}
            {\kern\fontdimen2\font}%
        }%
        
        % Allow breaks at special characters using \PYG... macros.
        \Wrappedbreaksatspecials
        % Breaks at punctuation characters . , ; ? ! and / need catcode=\active 	
        \OriginalVerbatim[#1,codes*=\Wrappedbreaksatpunct]%
    }
    \makeatother

    % Exact colors from NB
    \definecolor{incolor}{HTML}{303F9F}
    \definecolor{outcolor}{HTML}{D84315}
    \definecolor{cellborder}{HTML}{CFCFCF}
    \definecolor{cellbackground}{HTML}{F7F7F7}
    
    % prompt
    \makeatletter
    \newcommand{\boxspacing}{\kern\kvtcb@left@rule\kern\kvtcb@boxsep}
    \makeatother
    \newcommand{\prompt}[4]{
        {\ttfamily\llap{{\color{#2}[#3]:\hspace{3pt}#4}}\vspace{-\baselineskip}}
    }
    

    
    % Prevent overflowing lines due to hard-to-break entities
    \sloppy 
    % Setup hyperref package
    \hypersetup{
      breaklinks=true,  % so long urls are correctly broken across lines
      colorlinks=true,
      urlcolor=urlcolor,
      linkcolor=linkcolor,
      citecolor=citecolor,
      }
    % Slightly bigger margins than the latex defaults
    
    \geometry{verbose,tmargin=1in,bmargin=1in,lmargin=1in,rmargin=1in}
    
    

\begin{document}
    
    \maketitle
    
    

    
    \hypertarget{maze-on-fire}{%
\section{Maze on Fire}\label{maze-on-fire}}

Intro to AI Project 1

Reagan McFarland (rpm141), Alay Shah (acs286), Toshanraju Vysyaraju
(tv135)

Imports: - The only non standard import here is trange from tqdm, but
all uses of \texttt{trange} can be replaced with \texttt{range} with no
change in results - just no neat progress bar :)

    \begin{tcolorbox}[breakable, size=fbox, boxrule=1pt, pad at break*=1mm,colback=cellbackground, colframe=cellborder]
\prompt{In}{incolor}{ }{\boxspacing}
\begin{Verbatim}[commandchars=\\\{\}]
\PY{c+c1}{\PYZsh{} imports}
\PY{k+kn}{import} \PY{n+nn}{matplotlib}\PY{n+nn}{.}\PY{n+nn}{pyplot} \PY{k}{as} \PY{n+nn}{plt}
\PY{k+kn}{from} \PY{n+nn}{matplotlib} \PY{k+kn}{import} \PY{n}{cm}
\PY{k+kn}{from} \PY{n+nn}{matplotlib}\PY{n+nn}{.}\PY{n+nn}{colors} \PY{k+kn}{import} \PY{n}{ListedColormap}\PY{p}{,} \PY{n}{LinearSegmentedColormap}
\PY{k+kn}{from} \PY{n+nn}{copy} \PY{k+kn}{import} \PY{n}{deepcopy}
\PY{k+kn}{import} \PY{n+nn}{random}
\PY{k+kn}{import} \PY{n+nn}{numpy} \PY{k}{as} \PY{n+nn}{np}
\PY{k+kn}{from} \PY{n+nn}{tqdm}\PY{n+nn}{.}\PY{n+nn}{notebook} \PY{k+kn}{import} \PY{n}{trange}
\PY{k+kn}{from} \PY{n+nn}{tqdm}\PY{n+nn}{.}\PY{n+nn}{notebook} \PY{k+kn}{import} \PY{n}{tqdm}
\PY{k+kn}{from} \PY{n+nn}{queue} \PY{k+kn}{import} \PY{n}{PriorityQueue}\PY{p}{,} \PY{n}{Queue}
\PY{k+kn}{from} \PY{n+nn}{pprint} \PY{k+kn}{import} \PY{n}{pprint}
\PY{k+kn}{import} \PY{n+nn}{math}
\end{Verbatim}
\end{tcolorbox}

    \hypertarget{problem-1}{%
\subsection{Problem 1}\label{problem-1}}

\hypertarget{generating-a-maze}{%
\subsubsection{Generating a maze}\label{generating-a-maze}}

When thinking about how we were going to generate the maze, we first
wanted to figure out a good way to render the maze in
\texttt{jupyter\ notebooks}. This led us to color maps in
\texttt{matplotlib} using a custom color map, allowing us to mark nodes
certain colors based on the value. This resulted in us going with a 2D
list of boolean values, where: - \texttt{False} = open cell -
\texttt{True} = occupied cell

We also wanted to make it as easy as possible to reuse, so we have
optional params \texttt{p} and \texttt{dim}, representing the obstacle
density and dimension respecitively. We also have the optional params
\texttt{start} and \texttt{goal} as a tuple, that we can use to force
those spaces to be empty. These optional params get re-used in almost
every function. This is all done in the function \texttt{gen\_maze}.

    \hypertarget{rendering-a-maze}{%
\subsubsection{Rendering a maze}\label{rendering-a-maze}}

We also wanted a method to show a color map plot of our maze, and
settled on the following function after reading the \texttt{matplotlib}
docs.

Notice that we can pass in the index as a tuple for both the start and
end if needed. This is because we render these as gray instead of white
/ black making it easier to see the start and goal of the maze. These
are optional params though of course. This is done in the function
\texttt{render\_maze}.

    Putting these together, we can generate an example maze with
\texttt{p=0.3} and \texttt{dim=20}

    \begin{tcolorbox}[breakable, size=fbox, boxrule=1pt, pad at break*=1mm,colback=cellbackground, colframe=cellborder]
\prompt{In}{incolor}{ }{\boxspacing}
\begin{Verbatim}[commandchars=\\\{\}]
\PY{n}{example\PYZus{}maze} \PY{o}{=} \PY{n}{gen\PYZus{}maze}\PY{p}{(}\PY{p}{)}
\PY{c+c1}{\PYZsh{} example\PYZus{}maze2 = gen\PYZus{}maze(0.5, 50) \PYZsh{} we could also do this if we wanted to specify the object density and dimensions explicitly}
\PY{n}{render\PYZus{}maze}\PY{p}{(}\PY{n}{example\PYZus{}maze}\PY{p}{)}
\end{Verbatim}
\end{tcolorbox}

    \begin{center}
    \adjustimage{max size={0.9\linewidth}{0.9\paperheight}}{output_5_0.png}
    \end{center}
    { \hspace*{\fill} \\}
    
    \hypertarget{problem-2}{%
\subsection{Problem 2}\label{problem-2}}

\hypertarget{our-dfs-algorithm}{%
\subsubsection{Our DFS Algorithm}\label{our-dfs-algorithm}}

DFS is a very rudimentary search algorithm, with the only real decision
to be made when implementing is whether or not you want to a recursive
or iterative approach. We decided to go with a iterative approach just
because its easier to transform into BFS later. We also have optional
params here for our start and end cell indexes because our start and end
goals are pre-determined for most of our test cases, but can be changed
if needed with little alteration. We also have a traceNodes optional
parameter which will highlight then nodes visited whenever we pass the
maze to \texttt{render\_maze()}. All this is implemented in the function
\texttt{DFSUninformed}.

    \hypertarget{why-is-dfs-a-better-choice-than-bfs-here}{%
\subsubsection{Why is DFS a better choice than BFS
here?}\label{why-is-dfs-a-better-choice-than-bfs-here}}

DFS is a better choice than BFS here because we do not need to find an
optimal path. This allows us to save on the space complexity as our
fringe is exponentially smaller as we proved in class. In addition, we
know the goal node to be the furthest point from the start, i.e.~the
deepest part of the graph. DFS benefits greatly because it goes further
down one path rather than exploring all neighbors, like BFS. Therefore
as DFS searches for the `deepest' node, it is a much better option in
this scenario.

    \hypertarget{obstacle-density-p-vs.-probabilty-that-s-can-be-reached-from-g}{%
\subsubsection{\texorpdfstring{Obstacle Density \texttt{p}
vs.~Probabilty that \texttt{S} can be reached from
\texttt{G}}{Obstacle Density p vs.~Probabilty that S can be reached from G}}\label{obstacle-density-p-vs.-probabilty-that-s-can-be-reached-from-g}}

Our machine could handle dimension of 100 with 100 samples per object
density \texttt{p} we test. We are testing all \texttt{p} between
\texttt{0} and \texttt{1}, with a step of \texttt{0.1} for each
iteration.

    \hypertarget{problem-3}{%
\subsection{Problem 3}\label{problem-3}}

\hypertarget{our-bfs-algorithm}{%
\subsubsection{Our BFS Algorithm}\label{our-bfs-algorithm}}

BFS implements a queue instead of a stack in order to visit all
neighbors before visiting children. This is done in our function
\texttt{BFSUninformed} which returns if the goal is reached, the number
of visited cells, the length of the path to the goal, and the actual
path it took.

    Before we begin, we first wanted to define a function
\texttt{trace\_path} that would trace the optimal path so we could
visually see it on our graph. It will also return the length of the
optimal path and the actual path it self (in a stack).

    \hypertarget{our-a-algorithm}{%
\subsubsection{Our A* Algorithm}\label{our-a-algorithm}}

Before we implement our \texttt{A*} algorithm, we need to define a
function that can give us the euclidean distance between a cell and the
goal cell. This is done in the function \texttt{euclid}.

    With the heuristic defined, we can now write the A* algorithm in the
function \texttt{AStar} using almost the same functionality as BFS.
However, we are using a \texttt{PriorityQueue} instead of a normal
queue. The priority queue orders cells with a lower cost based on the
heuristic and actual cost from low to high.

    We can view a random example of how the number of nodes A* visits
vs.~BFS differs. In the below mazes, the grey represents cells that were
visited and the green represents the optimal path returned. The white
cells are the cells which are not visited. It is important to note that
there are multiple optimal paths and so the BFS path may not be
identical to the A* path but they both have identical length.

    \begin{tcolorbox}[breakable, size=fbox, boxrule=1pt, pad at break*=1mm,colback=cellbackground, colframe=cellborder]
\prompt{In}{incolor}{ }{\boxspacing}
\begin{Verbatim}[commandchars=\\\{\}]
\PY{n}{maze\PYZus{}bfs} \PY{o}{=} \PY{n}{gen\PYZus{}maze}\PY{p}{(}\PY{l+m+mf}{0.2}\PY{p}{,} \PY{l+m+mi}{200}\PY{p}{)}
\PY{n}{maze\PYZus{}astar} \PY{o}{=} \PY{n}{deepcopy}\PY{p}{(}\PY{n}{maze\PYZus{}bfs}\PY{p}{)}
\PY{n}{goal}\PY{o}{=}\PY{p}{(}\PY{n+nb}{len}\PY{p}{(}\PY{n}{maze\PYZus{}bfs}\PY{p}{)}\PY{o}{\PYZhy{}}\PY{l+m+mi}{1}\PY{p}{,} \PY{n+nb}{len}\PY{p}{(}\PY{n}{maze\PYZus{}bfs}\PY{p}{)}\PY{o}{\PYZhy{}}\PY{l+m+mi}{1}\PY{p}{)}
\PY{n}{bfs} \PY{o}{=} \PY{n}{BFSUninformed}\PY{p}{(}\PY{n}{maze\PYZus{}bfs}\PY{p}{,} \PY{n}{goal}\PY{o}{=}\PY{n}{goal}\PY{p}{,} \PY{n}{traceNodes}\PY{o}{=}\PY{k+kc}{True}\PY{p}{)}
\PY{n}{astar} \PY{o}{=} \PY{n}{AStar}\PY{p}{(}\PY{n}{maze\PYZus{}astar}\PY{p}{,}\PY{n}{goal}\PY{o}{=}\PY{n}{goal}\PY{p}{,} \PY{n}{traceNodes}\PY{o}{=}\PY{k+kc}{True}\PY{p}{)}
\PY{n+nb}{print}\PY{p}{(}\PY{l+s+s2}{\PYZdq{}}\PY{l+s+s2}{BFS Nodes Visited = }\PY{l+s+s2}{\PYZdq{}} \PY{o}{+} \PY{n+nb}{str}\PY{p}{(}\PY{n}{bfs}\PY{p}{[}\PY{l+m+mi}{0}\PY{p}{:}\PY{l+m+mi}{3}\PY{p}{]}\PY{p}{)}\PY{p}{)}
\PY{n}{render\PYZus{}maze}\PY{p}{(}\PY{n}{maze}\PY{o}{=}\PY{n}{maze\PYZus{}bfs}\PY{p}{,} \PY{n}{goal}\PY{o}{=}\PY{n}{goal}\PY{p}{,} \PY{n}{traceNodes}\PY{o}{=}\PY{k+kc}{True}\PY{p}{)}
\PY{n+nb}{print}\PY{p}{(}\PY{l+s+s2}{\PYZdq{}}\PY{l+s+s2}{A* Nodes Visited = }\PY{l+s+s2}{\PYZdq{}} \PY{o}{+} \PY{n+nb}{str}\PY{p}{(}\PY{n}{astar}\PY{p}{[}\PY{l+m+mi}{0}\PY{p}{:}\PY{l+m+mi}{3}\PY{p}{]}\PY{p}{)}\PY{p}{)}
\PY{n}{render\PYZus{}maze}\PY{p}{(}\PY{n}{maze}\PY{o}{=}\PY{n}{maze\PYZus{}astar}\PY{p}{,} \PY{n}{goal}\PY{o}{=}\PY{n}{goal}\PY{p}{,} \PY{n}{traceNodes}\PY{o}{=}\PY{k+kc}{True}\PY{p}{)}
\end{Verbatim}
\end{tcolorbox}

    \begin{Verbatim}[commandchars=\\\{\}]
BFS Nodes Visited = (True, 31817, 398)
A* Nodes Visited = (True, 30244, 398)
    \end{Verbatim}

    \begin{center}
    \adjustimage{max size={0.9\linewidth}{0.9\paperheight}}{output_14_1.png}
    \end{center}
    { \hspace*{\fill} \\}
    
    \begin{center}
    \adjustimage{max size={0.9\linewidth}{0.9\paperheight}}{output_14_2.png}
    \end{center}
    { \hspace*{\fill} \\}
    
    \hypertarget{number-of-nodes-explored-by-bfs---number-of-nodes-explored-by-a-vs.-obstacle-density-p}{%
\subsubsection{\texorpdfstring{Number of nodes explored by BFS - number
of nodes explored by A* vs.~obstacle density
\texttt{p}}{Number of nodes explored by BFS - number of nodes explored by A* vs.~obstacle density p}}\label{number-of-nodes-explored-by-bfs---number-of-nodes-explored-by-a-vs.-obstacle-density-p}}

In order to graph this, we are going to use the same assumption for the
last graph about sample sizes, steps, etc. On top of this, we are going
to create a function that for every single sample, will do the
following: 1. Generate a new maze 2. Run BFS and record the number of
nodes explored 3. Run A* and record the number of nodes explored

Then, at the end we will average these out across the steps and graph it
using matplotlib. First, let's create that function
\texttt{diff\_AStar\_BFS} that will do all the 3 steps above.

    Now, let's write some code to generate to generate the data for the
graph and then also render it, just like before.

    \begin{tcolorbox}[breakable, size=fbox, boxrule=1pt, pad at break*=1mm,colback=cellbackground, colframe=cellborder]
\prompt{In}{incolor}{ }{\boxspacing}
\begin{Verbatim}[commandchars=\\\{\}]
\PY{c+c1}{\PYZsh{} Settings}
\PY{n}{SAMPLE\PYZus{}COUNT} \PY{o}{=} \PY{l+m+mi}{100}
\PY{n}{STEP} \PY{o}{=} \PY{l+m+mf}{0.05}
\PY{n}{DIMENSION} \PY{o}{=} \PY{l+m+mi}{100}

\PY{c+c1}{\PYZsh{} Code to generate graph }
\PY{n}{densities} \PY{o}{=} \PY{n}{np}\PY{o}{.}\PY{n}{arange}\PY{p}{(}\PY{l+m+mi}{0}\PY{p}{,} \PY{l+m+mi}{1}\PY{p}{,} \PY{n}{STEP}\PY{p}{)}\PY{o}{.}\PY{n}{tolist}\PY{p}{(}\PY{p}{)}
\PY{n}{successes} \PY{o}{=} \PY{n+nb}{dict}\PY{p}{(}\PY{p}{)}
\PY{n}{density\PYZus{}count} \PY{o}{=} \PY{n+nb}{len}\PY{p}{(}\PY{n}{densities}\PY{p}{)}
\PY{k}{with} \PY{n}{tqdm}\PY{p}{(}\PY{n}{total}\PY{o}{=}\PY{n}{density\PYZus{}count} \PY{o}{*} \PY{n}{SAMPLE\PYZus{}COUNT}\PY{p}{)} \PY{k}{as} \PY{n}{pbar}\PY{p}{:}
    \PY{k}{for} \PY{n}{i} \PY{o+ow}{in} \PY{n+nb}{range}\PY{p}{(}\PY{n+nb}{len}\PY{p}{(}\PY{n}{densities}\PY{p}{)}\PY{p}{)}\PY{p}{:} 
        \PY{n}{p} \PY{o}{=} \PY{n}{densities}\PY{p}{[}\PY{n}{i}\PY{p}{]}
        \PY{k}{if} \PY{n}{p} \PY{o+ow}{not} \PY{o+ow}{in} \PY{n}{successes}\PY{p}{:}
            \PY{n}{successes}\PY{p}{[}\PY{n}{p}\PY{p}{]} \PY{o}{=} \PY{l+m+mi}{0}
        \PY{k}{for} \PY{n}{j} \PY{o+ow}{in} \PY{n+nb}{range}\PY{p}{(}\PY{n}{SAMPLE\PYZus{}COUNT}\PY{p}{)}\PY{p}{:}
            \PY{n}{successes}\PY{p}{[}\PY{n}{p}\PY{p}{]} \PY{o}{+}\PY{o}{=} \PY{n}{diff\PYZus{}AStar\PYZus{}BFS}\PY{p}{(}\PY{n}{p}\PY{p}{,} \PY{n}{DIMENSION}\PY{p}{)}
            \PY{n}{pbar}\PY{o}{.}\PY{n}{update}\PY{p}{(}\PY{l+m+mi}{1}\PY{p}{)}

\PY{n}{x\PYZus{}axis} \PY{o}{=} \PY{n}{densities}
\PY{n}{y\PYZus{}axis} \PY{o}{=} \PY{p}{[}\PY{n}{successes}\PY{p}{[}\PY{n}{x}\PY{p}{]} \PY{o}{/} \PY{n}{SAMPLE\PYZus{}COUNT} \PY{k}{for} \PY{n}{x} \PY{o+ow}{in} \PY{n}{densities}\PY{p}{]}

\PY{n}{plt}\PY{o}{.}\PY{n}{xlabel}\PY{p}{(}\PY{l+s+s2}{\PYZdq{}}\PY{l+s+s2}{obstacle density p}\PY{l+s+s2}{\PYZdq{}}\PY{p}{)}
\PY{n}{plt}\PY{o}{.}\PY{n}{ylabel}\PY{p}{(}\PY{l+s+s2}{\PYZdq{}}\PY{l+s+s2}{Nodes Visited BFS \PYZhy{} Nodes Visited A*}\PY{l+s+s2}{\PYZdq{}}\PY{p}{)}
\PY{n}{plt}\PY{o}{.}\PY{n}{title}\PY{p}{(}\PY{l+s+s2}{\PYZdq{}}\PY{l+s+s2}{BFS \PYZhy{} A* vs. object density p }\PY{l+s+se}{\PYZbs{}n}\PY{l+s+s2}{ (dim=}\PY{l+s+s2}{\PYZdq{}} \PY{o}{+} \PY{n+nb}{str}\PY{p}{(}\PY{n}{DIMENSION}\PY{p}{)} \PY{o}{+} \PY{l+s+s2}{\PYZdq{}}\PY{l+s+s2}{, samples=}\PY{l+s+s2}{\PYZdq{}} \PY{o}{+} \PY{n+nb}{str}\PY{p}{(}\PY{n}{SAMPLE\PYZus{}COUNT}\PY{p}{)} \PY{o}{+} \PY{l+s+s2}{\PYZdq{}}\PY{l+s+s2}{ with step=}\PY{l+s+s2}{\PYZdq{}} \PY{o}{+} \PY{n+nb}{str}\PY{p}{(}\PY{n}{STEP}\PY{p}{)} \PY{o}{+} \PY{l+s+s2}{\PYZdq{}}\PY{l+s+s2}{)}\PY{l+s+s2}{\PYZdq{}}\PY{p}{)}
\PY{n}{plt}\PY{o}{.}\PY{n}{plot}\PY{p}{(}\PY{n}{x\PYZus{}axis}\PY{p}{,} \PY{n}{y\PYZus{}axis}\PY{p}{)}
\end{Verbatim}
\end{tcolorbox}

    
    \begin{Verbatim}[commandchars=\\\{\}]
HBox(children=(FloatProgress(value=0.0, max=2000.0), HTML(value='')))
    \end{Verbatim}

    
    \begin{Verbatim}[commandchars=\\\{\}]

    \end{Verbatim}

            \begin{tcolorbox}[breakable, size=fbox, boxrule=.5pt, pad at break*=1mm, opacityfill=0]
\prompt{Out}{outcolor}{ }{\boxspacing}
\begin{Verbatim}[commandchars=\\\{\}]
[<matplotlib.lines.Line2D at 0x7fd6b1f15128>]
\end{Verbatim}
\end{tcolorbox}
        
    \begin{center}
    \adjustimage{max size={0.9\linewidth}{0.9\paperheight}}{output_17_3.png}
    \end{center}
    { \hspace*{\fill} \\}
    
    \hypertarget{if-there-is-no-path-from-s-to-g-what-should-this-difference-be}{%
\subsubsection{If there is no path from S to G, what should this
difference
be?}\label{if-there-is-no-path-from-s-to-g-what-should-this-difference-be}}

If there is no path, then the difference will be 0. This is because both
algorithms will have to check all the same nodes before it can 100\% be
sure that there is no path. Both algorithms are using a queue, its just
that A* will probably get there faster because its using a priority
queue with a heuristic. The algorithm cannot know that the path is
blocked and the heurestic is rendered useless, so in both cases, it will
explore each possibility in hopes of finding the goal node. Therefore,
the difference in the number of nodes traversed for each algorithm will
be the same when there is no path.

    \hypertarget{problem-4}{%
\subsection{Problem 4}\label{problem-4}}

    \hypertarget{whats-the-largest-dimension-you-can-solve-using-dfs-at-p0.3-in-less-than-a-minute}{%
\subsubsection{\texorpdfstring{What's the largest dimension you can
solve using DFS at \texttt{p=0.3} in less than a
minute?}{What's the largest dimension you can solve using DFS at p=0.3 in less than a minute?}}\label{whats-the-largest-dimension-you-can-solve-using-dfs-at-p0.3-in-less-than-a-minute}}

    \textbf{Through trial and error, we found that the largest maze with
density \texttt{p=0.3} that we can will solve is around 4350x4350}

    \hypertarget{whats-the-largest-dimension-you-can-solve-using-bfs-at-p0.3-in-less-than-a-minute}{%
\subsubsection{\texorpdfstring{What's the largest dimension you can
solve using BFS at \texttt{p=0.3} in less than a
minute?}{What's the largest dimension you can solve using BFS at p=0.3 in less than a minute?}}\label{whats-the-largest-dimension-you-can-solve-using-bfs-at-p0.3-in-less-than-a-minute}}

    \textbf{Through trial and error, we found that the largest maze with
density \texttt{p=0.3} that we can will solve is around 4500x4500}

    \hypertarget{whats-the-largest-dimension-you-can-solve-using-a-at-p0.3-in-less-than-a-minute}{%
\subsubsection{\texorpdfstring{What's the largest dimension you can
solve using A* at \texttt{p=0.3} in less than a
minute?}{What's the largest dimension you can solve using A* at p=0.3 in less than a minute?}}\label{whats-the-largest-dimension-you-can-solve-using-a-at-p0.3-in-less-than-a-minute}}

    \textbf{Through trial and error, we found that the largest maze with
density \texttt{p=0.3} that we can solve is around 15000x15000}

    \hypertarget{consider-as-you-solve-these-three-problems-simple-diagnostic-criteria-to-make-sure-you-areon-track.-the-path-returned-by-dfs-should-never-be-shorter-than-the-path-returned-by-bfs.the-path-returned-by-a-should-not-be-shorter-than-the-path-returned-by-bfs.-how-big-can-andshould-your-fringe-be-at-any-point-during-these-algorithms}{%
\subsubsection{Consider, as you solve these three problems, simple
diagnostic criteria to make sure you areon track. The path returned by
DFS should never be shorter than the path returned by BFS.The path
returned by A* should not be shorter than the path returned by BFS. How
big can andshould your fringe be at any point during these
algorithms?}\label{consider-as-you-solve-these-three-problems-simple-diagnostic-criteria-to-make-sure-you-areon-track.-the-path-returned-by-dfs-should-never-be-shorter-than-the-path-returned-by-bfs.the-path-returned-by-a-should-not-be-shorter-than-the-path-returned-by-bfs.-how-big-can-andshould-your-fringe-be-at-any-point-during-these-algorithms}}

In the maze, each node has at most 4 neighbors (up, down, left, right).
In class we derived the space complexity of the fringes. Here, we will
treat A* and BFS as the same, since we are dealing with worst case
(i.e.~\(h(n)\) is the same for all nodes). Let \(n\) be the size of the
maze. Then we arrive at a space complexity of \(O(4n)\) for DFS.
However, in the case of BFS and A*, the space complexity of our fringe
will be \(O(4^n)\). At runtime, the A* algorithm might have a fringe
smaller than \(O(4^n)\) as it only explores the nodes that bring it
closer to the goal node, but ultimately it is also bounded by a space
complexity of \(O(4^n)\).

    \hypertarget{part-2-maze-on-fire}{%
\section{Part 2 Maze on Fire}\label{part-2-maze-on-fire}}

Looking at non-static mazes, there is a fire that is acitvely burning
down and we need to get out before running into the fire. Solving for
the current state may not work for future states of the maze.

    \hypertarget{generating-a-maze}{%
\subsection{Generating a maze}\label{generating-a-maze}}

We decided to re-use our \texttt{gen\_maze(dim,\ p)} function to create
the maze and then choose a random cell to be on fire. To ensure that the
goal and start nodes are not the starting points of the fire, we used a
function \texttt{randomFirestart(dim)} to properly pick the starting
point of the fire. We then use the method \texttt{gen\_fireMaze} to
generate a maze with a random fire start cell.

    \hypertarget{expanding-the-fire}{%
\subsubsection{Expanding the fire}\label{expanding-the-fire}}

We create a matrix that houses the current state of the fire, and based
on that will determine which cells will be on fire in the method
\texttt{expandFireOneStep}, using the stated parameters:

\begin{itemize}
\tightlist
\item
  If a free cell has no burning neighbors, it will still be free in the
  next time step.
\item
  If a cell is on fire, it will still be on fire in the next time step.
\item
  A blocked cell cannot catch on fire.
\item
  If a free cell has \(k\) burning neighbors, it will be on fire in the
  next time step with probability \(1 − (1 − q)^k\)
\end{itemize}

    We can illustrate this function by rendering maze with
\texttt{dim=20,\ p=0.3\ ,}and\texttt{q=0.3}, and having the fire expand
10 steps.

    \begin{tcolorbox}[breakable, size=fbox, boxrule=1pt, pad at break*=1mm,colback=cellbackground, colframe=cellborder]
\prompt{In}{incolor}{ }{\boxspacing}
\begin{Verbatim}[commandchars=\\\{\}]
\PY{n}{maze}\PY{p}{,}\PY{n}{\PYZus{}} \PY{o}{=} \PY{n}{gen\PYZus{}fireMaze}\PY{p}{(}\PY{n}{dim}\PY{o}{=}\PY{l+m+mi}{20}\PY{p}{,} \PY{n}{p}\PY{o}{=}\PY{l+m+mf}{0.3}\PY{p}{)}
\PY{n}{render\PYZus{}maze}\PY{p}{(}\PY{n}{maze}\PY{p}{,}\PY{n}{goal}\PY{o}{=}\PY{p}{(}\PY{l+m+mi}{19}\PY{p}{,}\PY{l+m+mi}{19}\PY{p}{)}\PY{p}{,}\PY{n}{fire}\PY{o}{=}\PY{k+kc}{True}\PY{p}{)}
\PY{k}{for} \PY{n}{i} \PY{o+ow}{in} \PY{n+nb}{range}\PY{p}{(}\PY{l+m+mi}{10}\PY{p}{)}\PY{p}{:}
    \PY{n}{maze} \PY{o}{=} \PY{n}{expandFireOneStep}\PY{p}{(}\PY{n}{maze}\PY{p}{,} \PY{l+m+mf}{0.3}\PY{p}{)}
\PY{n}{render\PYZus{}maze}\PY{p}{(}\PY{n}{maze}\PY{p}{,}\PY{n}{goal}\PY{o}{=}\PY{p}{(}\PY{l+m+mi}{19}\PY{p}{,}\PY{l+m+mi}{19}\PY{p}{)}\PY{p}{,}\PY{n}{fire}\PY{o}{=}\PY{k+kc}{True}\PY{p}{)} 
\end{Verbatim}
\end{tcolorbox}

    \begin{center}
    \adjustimage{max size={0.9\linewidth}{0.9\paperheight}}{output_31_0.png}
    \end{center}
    { \hspace*{\fill} \\}
    
    \begin{center}
    \adjustimage{max size={0.9\linewidth}{0.9\paperheight}}{output_31_1.png}
    \end{center}
    { \hspace*{\fill} \\}
    
    \hypertarget{strategy-1-no-strategy}{%
\subsection{Strategy 1, No Strategy}\label{strategy-1-no-strategy}}

We will use A* to calculate the shortest path to the goal and follow as
if there was no fire in the method \texttt{strat\_one}. This strategy
blindly follows the path that A* returns and does not account for the
current or future states of the fire.

    With strategy 1 now well defined, let us create a method
\texttt{test\_strat1} to test it easily for a given dimension and fire
spreading rate. For each test, we will generate the fire start position
randomly 10 times and return the number of successes encountered.

    Now with our testing method \texttt{test\_strat1} defined above, we can
average out probability of success over a defined interval and graph the
results. In the graph shown below, we ran Strategy 1 with
\texttt{dimension\ 20}, \texttt{sample\ count\ 20}, and increased
\texttt{q} by \texttt{0.1}.

    \begin{tcolorbox}[breakable, size=fbox, boxrule=1pt, pad at break*=1mm,colback=cellbackground, colframe=cellborder]
\prompt{In}{incolor}{ }{\boxspacing}
\begin{Verbatim}[commandchars=\\\{\}]
\PY{c+c1}{\PYZsh{} Settings}
\PY{n}{SAMPLE\PYZus{}COUNT} \PY{o}{=} \PY{l+m+mi}{20}
\PY{n}{STEP} \PY{o}{=} \PY{l+m+mf}{0.1}
\PY{n}{DIMENSION} \PY{o}{=} \PY{l+m+mi}{20}

\PY{n}{successes} \PY{o}{=} \PY{n+nb}{dict}\PY{p}{(}\PY{p}{)}

\PY{c+c1}{\PYZsh{} Code to generate graph }
\PY{n}{densities} \PY{o}{=} \PY{n}{np}\PY{o}{.}\PY{n}{arange}\PY{p}{(}\PY{l+m+mi}{0}\PY{p}{,} \PY{l+m+mi}{1}\PY{p}{,} \PY{n}{STEP}\PY{p}{)}\PY{o}{.}\PY{n}{tolist}\PY{p}{(}\PY{p}{)}
\PY{n}{density\PYZus{}count} \PY{o}{=} \PY{n+nb}{len}\PY{p}{(}\PY{n}{densities}\PY{p}{)}
\PY{k}{with} \PY{n}{tqdm}\PY{p}{(}\PY{n}{total}\PY{o}{=}\PY{n}{density\PYZus{}count} \PY{o}{*} \PY{n}{SAMPLE\PYZus{}COUNT}\PY{p}{)} \PY{k}{as} \PY{n}{pbar}\PY{p}{:}
    \PY{k}{for} \PY{n}{i} \PY{o+ow}{in} \PY{n+nb}{range}\PY{p}{(}\PY{n+nb}{len}\PY{p}{(}\PY{n}{densities}\PY{p}{)}\PY{p}{)}\PY{p}{:} 
        \PY{n}{q} \PY{o}{=} \PY{n}{densities}\PY{p}{[}\PY{n}{i}\PY{p}{]}
        \PY{k}{if} \PY{n}{q} \PY{o+ow}{not} \PY{o+ow}{in} \PY{n}{successes}\PY{p}{:}
            \PY{n}{successes}\PY{p}{[}\PY{n}{q}\PY{p}{]} \PY{o}{=} \PY{l+m+mi}{0}
        \PY{k}{for} \PY{n}{j} \PY{o+ow}{in} \PY{n+nb}{range}\PY{p}{(}\PY{n}{SAMPLE\PYZus{}COUNT}\PY{p}{)}\PY{p}{:}
            \PY{n}{successes}\PY{p}{[}\PY{n}{q}\PY{p}{]} \PY{o}{+}\PY{o}{=} \PY{n}{test\PYZus{}strat1}\PY{p}{(}\PY{n}{dim}\PY{o}{=}\PY{n}{DIMENSION}\PY{p}{,} \PY{n}{q}\PY{o}{=}\PY{n}{q}\PY{p}{)}
            \PY{n}{pbar}\PY{o}{.}\PY{n}{update}\PY{p}{(}\PY{l+m+mi}{1}\PY{p}{)}

\PY{n}{x\PYZus{}axis} \PY{o}{=} \PY{n}{densities}
\PY{n}{strat\PYZus{}1\PYZus{}y\PYZus{}axis} \PY{o}{=} \PY{p}{[}\PY{n}{successes}\PY{p}{[}\PY{n}{x}\PY{p}{]} \PY{o}{/} \PY{p}{(}\PY{n}{SAMPLE\PYZus{}COUNT} \PY{o}{*} \PY{l+m+mi}{10}\PY{p}{)} \PY{k}{for} \PY{n}{x} \PY{o+ow}{in} \PY{n}{densities}\PY{p}{]}

\PY{n}{plt}\PY{o}{.}\PY{n}{xlabel}\PY{p}{(}\PY{l+s+s2}{\PYZdq{}}\PY{l+s+s2}{fire spreadibility q}\PY{l+s+s2}{\PYZdq{}}\PY{p}{)}
\PY{n}{plt}\PY{o}{.}\PY{n}{ylabel}\PY{p}{(}\PY{l+s+s2}{\PYZdq{}}\PY{l+s+s2}{Probability of Success}\PY{l+s+s2}{\PYZdq{}}\PY{p}{)}
\PY{n}{plt}\PY{o}{.}\PY{n}{title}\PY{p}{(}\PY{l+s+s2}{\PYZdq{}}\PY{l+s+s2}{Strategy 1 vs q }\PY{l+s+se}{\PYZbs{}n}\PY{l+s+s2}{ (dim=}\PY{l+s+s2}{\PYZdq{}} \PY{o}{+} \PY{n+nb}{str}\PY{p}{(}\PY{n}{DIMENSION}\PY{p}{)} \PY{o}{+} \PY{l+s+s2}{\PYZdq{}}\PY{l+s+s2}{, samples=}\PY{l+s+s2}{\PYZdq{}} \PY{o}{+} \PY{n+nb}{str}\PY{p}{(}\PY{n}{SAMPLE\PYZus{}COUNT}\PY{p}{)} \PY{o}{+} \PY{l+s+s2}{\PYZdq{}}\PY{l+s+s2}{ with step=}\PY{l+s+s2}{\PYZdq{}} \PY{o}{+} \PY{n+nb}{str}\PY{p}{(}\PY{n}{STEP}\PY{p}{)} \PY{o}{+} \PY{l+s+s2}{\PYZdq{}}\PY{l+s+s2}{)}\PY{l+s+s2}{\PYZdq{}}\PY{p}{)}
\PY{n}{plt}\PY{o}{.}\PY{n}{plot}\PY{p}{(}\PY{n}{x\PYZus{}axis}\PY{p}{,} \PY{n}{strat\PYZus{}1\PYZus{}y\PYZus{}axis}\PY{p}{)}
\end{Verbatim}
\end{tcolorbox}

    
    \begin{Verbatim}[commandchars=\\\{\}]
HBox(children=(FloatProgress(value=0.0, max=200.0), HTML(value='')))
    \end{Verbatim}

    
    \begin{Verbatim}[commandchars=\\\{\}]

    \end{Verbatim}

            \begin{tcolorbox}[breakable, size=fbox, boxrule=.5pt, pad at break*=1mm, opacityfill=0]
\prompt{Out}{outcolor}{ }{\boxspacing}
\begin{Verbatim}[commandchars=\\\{\}]
[<matplotlib.lines.Line2D at 0x7fee2e2ee518>]
\end{Verbatim}
\end{tcolorbox}
        
    \begin{center}
    \adjustimage{max size={0.9\linewidth}{0.9\paperheight}}{output_35_3.png}
    \end{center}
    { \hspace*{\fill} \\}
    
    \hypertarget{strategy-2-recompute-path-at-every-step}{%
\subsection{Strategy 2, Recompute Path at Every
Step}\label{strategy-2-recompute-path-at-every-step}}

Strategy 2 is very similar to Strategy 1 but we recalculate our A* path
at every step making sure to update our path to take into account the
changes in our fire. This is done in the method \texttt{strat\_two}.

    With strategy 2 now well defined, let us create a method
\texttt{test\_strat2} to test it easily for a given dimension and fire
spreading rate. For each test, we will generate the fire start position
randomly 10 times and return the number of successes encountered.

    Now with our testing method \texttt{test\_strat2} defined, we can
average out probability of success over a defined interval and graph the
results. In the graph shown below, we ran Strategy 2 with
\texttt{dimension\ 20}, \texttt{sample\ count\ 20}, and increased
\texttt{q} by \texttt{0.1}.

    \begin{tcolorbox}[breakable, size=fbox, boxrule=1pt, pad at break*=1mm,colback=cellbackground, colframe=cellborder]
\prompt{In}{incolor}{ }{\boxspacing}
\begin{Verbatim}[commandchars=\\\{\}]
\PY{c+c1}{\PYZsh{} Settings}
\PY{n}{SAMPLE\PYZus{}COUNT} \PY{o}{=} \PY{l+m+mi}{20}
\PY{n}{STEP} \PY{o}{=} \PY{l+m+mf}{0.1}
\PY{n}{DIMENSION} \PY{o}{=} \PY{l+m+mi}{20}

\PY{n}{successes} \PY{o}{=} \PY{n+nb}{dict}\PY{p}{(}\PY{p}{)}

\PY{c+c1}{\PYZsh{} Code to generate graph }
\PY{n}{densities} \PY{o}{=} \PY{n}{np}\PY{o}{.}\PY{n}{arange}\PY{p}{(}\PY{l+m+mi}{0}\PY{p}{,} \PY{l+m+mi}{1}\PY{p}{,} \PY{n}{STEP}\PY{p}{)}\PY{o}{.}\PY{n}{tolist}\PY{p}{(}\PY{p}{)}
\PY{n}{density\PYZus{}count} \PY{o}{=} \PY{n+nb}{len}\PY{p}{(}\PY{n}{densities}\PY{p}{)}
\PY{k}{with} \PY{n}{tqdm}\PY{p}{(}\PY{n}{total}\PY{o}{=}\PY{n}{density\PYZus{}count} \PY{o}{*} \PY{n}{SAMPLE\PYZus{}COUNT}\PY{p}{)} \PY{k}{as} \PY{n}{pbar}\PY{p}{:}
    \PY{k}{for} \PY{n}{i} \PY{o+ow}{in} \PY{n+nb}{range}\PY{p}{(}\PY{n+nb}{len}\PY{p}{(}\PY{n}{densities}\PY{p}{)}\PY{p}{)}\PY{p}{:} 
        \PY{n}{q} \PY{o}{=} \PY{n}{densities}\PY{p}{[}\PY{n}{i}\PY{p}{]}
        \PY{k}{if} \PY{n}{q} \PY{o+ow}{not} \PY{o+ow}{in} \PY{n}{successes}\PY{p}{:}
            \PY{n}{successes}\PY{p}{[}\PY{n}{q}\PY{p}{]} \PY{o}{=} \PY{l+m+mi}{0}
        \PY{k}{for} \PY{n}{j} \PY{o+ow}{in} \PY{n+nb}{range}\PY{p}{(}\PY{n}{SAMPLE\PYZus{}COUNT}\PY{p}{)}\PY{p}{:}
            \PY{n}{successes}\PY{p}{[}\PY{n}{q}\PY{p}{]} \PY{o}{+}\PY{o}{=} \PY{n}{test\PYZus{}strat2}\PY{p}{(}\PY{n}{dim}\PY{o}{=}\PY{n}{DIMENSION}\PY{p}{,} \PY{n}{q}\PY{o}{=}\PY{n}{q}\PY{p}{)}
            \PY{n}{pbar}\PY{o}{.}\PY{n}{update}\PY{p}{(}\PY{l+m+mi}{1}\PY{p}{)}

\PY{n}{x\PYZus{}axis} \PY{o}{=} \PY{n}{densities}
\PY{n}{strat\PYZus{}2\PYZus{}y\PYZus{}axis} \PY{o}{=} \PY{p}{[}\PY{n}{successes}\PY{p}{[}\PY{n}{x}\PY{p}{]} \PY{o}{/} \PY{p}{(}\PY{n}{SAMPLE\PYZus{}COUNT} \PY{o}{*} \PY{l+m+mi}{10}\PY{p}{)} \PY{k}{for} \PY{n}{x} \PY{o+ow}{in} \PY{n}{densities}\PY{p}{]}

\PY{n}{plt}\PY{o}{.}\PY{n}{xlabel}\PY{p}{(}\PY{l+s+s2}{\PYZdq{}}\PY{l+s+s2}{fire spreadibility q}\PY{l+s+s2}{\PYZdq{}}\PY{p}{)}
\PY{n}{plt}\PY{o}{.}\PY{n}{ylabel}\PY{p}{(}\PY{l+s+s2}{\PYZdq{}}\PY{l+s+s2}{Probability of Success}\PY{l+s+s2}{\PYZdq{}}\PY{p}{)}
\PY{n}{plt}\PY{o}{.}\PY{n}{title}\PY{p}{(}\PY{l+s+s2}{\PYZdq{}}\PY{l+s+s2}{Strategy 2 vs q }\PY{l+s+se}{\PYZbs{}n}\PY{l+s+s2}{ (dim=}\PY{l+s+s2}{\PYZdq{}} \PY{o}{+} \PY{n+nb}{str}\PY{p}{(}\PY{n}{DIMENSION}\PY{p}{)} \PY{o}{+} \PY{l+s+s2}{\PYZdq{}}\PY{l+s+s2}{, samples=}\PY{l+s+s2}{\PYZdq{}} \PY{o}{+} \PY{n+nb}{str}\PY{p}{(}\PY{n}{SAMPLE\PYZus{}COUNT}\PY{p}{)} \PY{o}{+} \PY{l+s+s2}{\PYZdq{}}\PY{l+s+s2}{ with step=}\PY{l+s+s2}{\PYZdq{}} \PY{o}{+} \PY{n+nb}{str}\PY{p}{(}\PY{n}{STEP}\PY{p}{)} \PY{o}{+} \PY{l+s+s2}{\PYZdq{}}\PY{l+s+s2}{)}\PY{l+s+s2}{\PYZdq{}}\PY{p}{)}
\PY{n}{plt}\PY{o}{.}\PY{n}{plot}\PY{p}{(}\PY{n}{x\PYZus{}axis}\PY{p}{,} \PY{n}{strat\PYZus{}2\PYZus{}y\PYZus{}axis}\PY{p}{)}
\end{Verbatim}
\end{tcolorbox}

    
    \begin{Verbatim}[commandchars=\\\{\}]
HBox(children=(FloatProgress(value=0.0, max=200.0), HTML(value='')))
    \end{Verbatim}

    
    \begin{Verbatim}[commandchars=\\\{\}]

    \end{Verbatim}

            \begin{tcolorbox}[breakable, size=fbox, boxrule=.5pt, pad at break*=1mm, opacityfill=0]
\prompt{Out}{outcolor}{ }{\boxspacing}
\begin{Verbatim}[commandchars=\\\{\}]
[<matplotlib.lines.Line2D at 0x7fd6b350a5f8>]
\end{Verbatim}
\end{tcolorbox}
        
    \begin{center}
    \adjustimage{max size={0.9\linewidth}{0.9\paperheight}}{output_39_3.png}
    \end{center}
    { \hspace*{\fill} \\}
    
    \hypertarget{problem-5---strategy-3-future-risk-adjusted-path-frap}{%
\subsection{Problem 5 - Strategy 3, Future Risk Adjusted Path
(FRAP)}\label{problem-5---strategy-3-future-risk-adjusted-path-frap}}

    \hypertarget{describe-your-improved-strategy-3.-how-does-it-account-for-the-unknown-future}{%
\subsubsection{Describe your improved Strategy 3. How does it account
for the unknown
future?}\label{describe-your-improved-strategy-3.-how-does-it-account-for-the-unknown-future}}

    Strategy 3 is our way to solve this maze on fire problem. The problem
with Strategy 2 is that it fails to take into account the future state
of the maze, resulting in sub-optimal results. Unlike Strategy 1 and
Strategy 2, our Strategy 3 takes into account not only the current state
of the fire but also potential future states of the fire. To do this, we
generate a fire map that takes in the current state of the maze,
including what is on fire currently. From this, assuming a worst case
\texttt{q\ =\ 1.0} fire spread probability, we calculate at what step
each cell will catch on fire and is therefore a representation of the
future of the fire. This is then used to create a weighting for each
cell. This weighting is based on the cell's proximity to the goal as
well as at which step it will be on fire in the worst case (based on the
fire map) and at which step the cell will be visited by our agent. These
weights are used as the heuristics for A*. More specifically, the cost
of the cell in the fringe is equal to the manhattan distance to the goal
+ the cost to reach the goal - the cell's value in the fire map.

The first step in this process is to create a list, mapping a tuple of
x,y coordinates to the worst case fire expansion at a given cell. What
we mean by this is that if the fire starts at some arbitrary cell
\texttt{C}, we can assume the worst conditions for the fire
(\texttt{q\ =\ 1.0}), and calculate for each cell in the grid how many
steps it would take for it to reach it. To implement this, we will
define a function \texttt{generate\_fire\_step\_maze}, which will
generate and return this list for a given maze.

    We are also going to want a way to render this fire maze for
visualization purposes, which is defined by the function
\texttt{render\_fire\_maze}. The darker the cell is, the earlier it is
on fire (in the worst case of \texttt{q\ =\ 1.0}) and the lighter the
cell is, the later it is on fire.

    We can take a look at a visualization of what we are talking about here,
by rendering a view of a generated maze and its corresponding
\texttt{fire\_step\_map}.

    \begin{tcolorbox}[breakable, size=fbox, boxrule=1pt, pad at break*=1mm,colback=cellbackground, colframe=cellborder]
\prompt{In}{incolor}{ }{\boxspacing}
\begin{Verbatim}[commandchars=\\\{\}]
\PY{n}{maze}\PY{p}{,} \PY{n}{fire} \PY{o}{=} \PY{n}{gen\PYZus{}fireMaze}\PY{p}{(}\PY{l+m+mi}{20}\PY{p}{,} \PY{l+m+mf}{0.3}\PY{p}{)}
\PY{n}{fire\PYZus{}step\PYZus{}map}\PY{p}{,} \PY{n}{fire\PYZus{}step\PYZus{}maze} \PY{o}{=} \PY{n}{generate\PYZus{}fire\PYZus{}step\PYZus{}maze}\PY{p}{(}\PY{n}{maze}\PY{p}{)}
\PY{n}{render\PYZus{}maze}\PY{p}{(}\PY{n}{maze}\PY{p}{,}\PY{n}{goal}\PY{o}{=}\PY{p}{(}\PY{l+m+mi}{19}\PY{p}{,}\PY{l+m+mi}{19}\PY{p}{)}\PY{p}{,} \PY{n}{fire}\PY{o}{=}\PY{k+kc}{True}\PY{p}{)}
\PY{n}{render\PYZus{}fire\PYZus{}maze}\PY{p}{(}\PY{n}{fire\PYZus{}step\PYZus{}maze}\PY{p}{,} \PY{n}{showValues}\PY{o}{=}\PY{k+kc}{True}\PY{p}{)}
\end{Verbatim}
\end{tcolorbox}

    \begin{center}
    \adjustimage{max size={0.9\linewidth}{0.9\paperheight}}{output_45_0.png}
    \end{center}
    { \hspace*{\fill} \\}
    
    \begin{center}
    \adjustimage{max size={0.9\linewidth}{0.9\paperheight}}{output_45_1.png}
    \end{center}
    { \hspace*{\fill} \\}
    
    
    \begin{Verbatim}[commandchars=\\\{\}]
<Figure size 600x400 with 0 Axes>
    \end{Verbatim}

    
    At this point we want to modify our A* in order to account for the
\texttt{fire\_step\_map}, i.e the future states of the maze. In
addition, we will use the manhattan distance to weigh the nodes, this
will allow us to explore optimal paths further away from the diagonal.
We do all this in the functions \texttt{manHat} and \texttt{AStar\_mod}.

    Now we can implement our Strategy 3 in the function \texttt{strat\_3}.

    With strategy 3 now well defined, let us create a method
\texttt{test\_strat3} to test it easily for a given dimension and fire
spreading rate. For each test, we will generate the fire start position
randomly 10 times and return the number of successes encountered. With
our testing method \texttt{test\_strat3} defined, we can average out
probability of success over a defined interval.

    \hypertarget{problem-6}{%
\subsection{Problem 6}\label{problem-6}}

    \hypertarget{plot-for-strategy-1-2-and-3-a-graph-of-average-strategy-success-rate-vs-flammability-q-at-p-0.3.-where-do-the-different-strategies-perform-the-same-where-do-they-perform-differently-why}{%
\subsubsection{Plot, for Strategy 1, 2, and 3, a graph of `average
strategy success rate' vs `flammability q' at p = 0.3. Where do the
different strategies perform the same? Where do they perform
differently?
Why?}\label{plot-for-strategy-1-2-and-3-a-graph-of-average-strategy-success-rate-vs-flammability-q-at-p-0.3.-where-do-the-different-strategies-perform-the-same-where-do-they-perform-differently-why}}

    We can compare each strategy against each other by plotting them on the
same graph.

    \begin{tcolorbox}[breakable, size=fbox, boxrule=1pt, pad at break*=1mm,colback=cellbackground, colframe=cellborder]
\prompt{In}{incolor}{ }{\boxspacing}
\begin{Verbatim}[commandchars=\\\{\}]
\PY{c+c1}{\PYZsh{} Settings}
\PY{n}{SAMPLE\PYZus{}COUNT} \PY{o}{=} \PY{l+m+mi}{20}
\PY{n}{STEP} \PY{o}{=} \PY{l+m+mf}{0.05}
\PY{n}{DIMENSION} \PY{o}{=} \PY{l+m+mi}{40}

\PY{c+c1}{\PYZsh{} Code to generate graph }
\PY{n}{densities} \PY{o}{=} \PY{n}{np}\PY{o}{.}\PY{n}{arange}\PY{p}{(}\PY{l+m+mf}{0.0}\PY{p}{,} \PY{l+m+mi}{1}\PY{p}{,} \PY{n}{STEP}\PY{p}{)}\PY{o}{.}\PY{n}{tolist}\PY{p}{(}\PY{p}{)}
\PY{n}{successes1} \PY{o}{=} \PY{n+nb}{dict}\PY{p}{(}\PY{p}{)}
\PY{n}{successes2} \PY{o}{=} \PY{n+nb}{dict}\PY{p}{(}\PY{p}{)}
\PY{n}{successes3} \PY{o}{=} \PY{n+nb}{dict}\PY{p}{(}\PY{p}{)}
\PY{n}{density\PYZus{}count} \PY{o}{=} \PY{n+nb}{len}\PY{p}{(}\PY{n}{densities}\PY{p}{)}
\PY{k}{with} \PY{n}{tqdm}\PY{p}{(}\PY{n}{total}\PY{o}{=}\PY{n}{density\PYZus{}count} \PY{o}{*} \PY{n}{SAMPLE\PYZus{}COUNT}\PY{p}{)} \PY{k}{as} \PY{n}{pbar}\PY{p}{:}
    \PY{k}{for} \PY{n}{i} \PY{o+ow}{in} \PY{n+nb}{range}\PY{p}{(}\PY{n+nb}{len}\PY{p}{(}\PY{n}{densities}\PY{p}{)}\PY{p}{)}\PY{p}{:} 
        \PY{n}{q} \PY{o}{=} \PY{n}{densities}\PY{p}{[}\PY{n}{i}\PY{p}{]}
        \PY{k}{if} \PY{n}{q} \PY{o+ow}{not} \PY{o+ow}{in} \PY{n}{successes2}\PY{p}{:}
            \PY{n}{successes1}\PY{p}{[}\PY{n}{q}\PY{p}{]} \PY{o}{=} \PY{l+m+mi}{0}
            \PY{n}{successes2}\PY{p}{[}\PY{n}{q}\PY{p}{]} \PY{o}{=} \PY{l+m+mi}{0}
            \PY{n}{successes3}\PY{p}{[}\PY{n}{q}\PY{p}{]} \PY{o}{=} \PY{l+m+mi}{0}
        \PY{k}{for} \PY{n}{j} \PY{o+ow}{in} \PY{n+nb}{range}\PY{p}{(}\PY{n}{SAMPLE\PYZus{}COUNT}\PY{p}{)}\PY{p}{:}
            \PY{n}{successes1}\PY{p}{[}\PY{n}{q}\PY{p}{]}\PY{o}{+}\PY{o}{=} \PY{n}{test\PYZus{}strat1}\PY{p}{(}\PY{n}{dim}\PY{o}{=}\PY{n}{DIMENSION}\PY{p}{,} \PY{n}{q}\PY{o}{=}\PY{n}{q}\PY{p}{)}
            \PY{n}{successes2}\PY{p}{[}\PY{n}{q}\PY{p}{]} \PY{o}{+}\PY{o}{=} \PY{n}{test\PYZus{}strat2}\PY{p}{(}\PY{n}{dim}\PY{o}{=}\PY{n}{DIMENSION}\PY{p}{,} \PY{n}{q}\PY{o}{=}\PY{n}{q}\PY{p}{)}
            \PY{n}{successes3}\PY{p}{[}\PY{n}{q}\PY{p}{]} \PY{o}{+}\PY{o}{=} \PY{n}{test\PYZus{}strat3}\PY{p}{(}\PY{n}{dim}\PY{o}{=}\PY{n}{DIMENSION}\PY{p}{,} \PY{n}{q}\PY{o}{=}\PY{n}{q}\PY{p}{)}
            \PY{n}{pbar}\PY{o}{.}\PY{n}{update}\PY{p}{(}\PY{l+m+mi}{1}\PY{p}{)}

\PY{n}{x\PYZus{}axis} \PY{o}{=} \PY{n}{densities}
\PY{n}{y\PYZus{}axis} \PY{o}{=} \PY{p}{[}\PY{n}{successes3}\PY{p}{[}\PY{n}{x}\PY{p}{]} \PY{o}{/} \PY{p}{(}\PY{n}{SAMPLE\PYZus{}COUNT}\PY{o}{*}\PY{l+m+mi}{10}\PY{p}{)} \PY{k}{for} \PY{n}{x} \PY{o+ow}{in} \PY{n}{densities}\PY{p}{]}
\PY{n}{y\PYZus{}2} \PY{o}{=} \PY{p}{[}\PY{n}{successes2}\PY{p}{[}\PY{n}{x}\PY{p}{]}\PY{o}{/}\PY{p}{(}\PY{n}{SAMPLE\PYZus{}COUNT}\PY{o}{*}\PY{l+m+mi}{10}\PY{p}{)} \PY{k}{for} \PY{n}{x} \PY{o+ow}{in} \PY{n}{densities}\PY{p}{]}
\PY{n}{y\PYZus{}1} \PY{o}{=} \PY{p}{[}\PY{n}{successes1}\PY{p}{[}\PY{n}{x}\PY{p}{]}\PY{o}{/}\PY{p}{(}\PY{n}{SAMPLE\PYZus{}COUNT}\PY{o}{*}\PY{l+m+mi}{10}\PY{p}{)} \PY{k}{for} \PY{n}{x} \PY{o+ow}{in} \PY{n}{densities}\PY{p}{]}

\PY{n}{plt}\PY{o}{.}\PY{n}{xlabel}\PY{p}{(}\PY{l+s+s2}{\PYZdq{}}\PY{l+s+s2}{fire spreadibility q}\PY{l+s+s2}{\PYZdq{}}\PY{p}{)}
\PY{n}{plt}\PY{o}{.}\PY{n}{ylabel}\PY{p}{(}\PY{l+s+s2}{\PYZdq{}}\PY{l+s+s2}{Probability of Success}\PY{l+s+s2}{\PYZdq{}}\PY{p}{)}
\PY{n}{plt}\PY{o}{.}\PY{n}{title}\PY{p}{(}\PY{l+s+s2}{\PYZdq{}}\PY{l+s+s2}{Average Strategy Success vs flammability q (at p=0.3) }\PY{l+s+se}{\PYZbs{}n}\PY{l+s+s2}{ (dim=}\PY{l+s+s2}{\PYZdq{}} \PY{o}{+} \PY{n+nb}{str}\PY{p}{(}\PY{n}{DIMENSION}\PY{p}{)} \PY{o}{+} \PY{l+s+s2}{\PYZdq{}}\PY{l+s+s2}{, samples=}\PY{l+s+s2}{\PYZdq{}} \PY{o}{+} \PY{n+nb}{str}\PY{p}{(}\PY{n}{SAMPLE\PYZus{}COUNT}\PY{p}{)} \PY{o}{+} \PY{l+s+s2}{\PYZdq{}}\PY{l+s+s2}{ with step=}\PY{l+s+s2}{\PYZdq{}} \PY{o}{+} \PY{n+nb}{str}\PY{p}{(}\PY{n}{STEP}\PY{p}{)} \PY{o}{+} \PY{l+s+s2}{\PYZdq{}}\PY{l+s+s2}{)}\PY{l+s+s2}{\PYZdq{}}\PY{p}{)}
\PY{n}{plt}\PY{o}{.}\PY{n}{plot}\PY{p}{(}\PY{n}{x\PYZus{}axis}\PY{p}{,} \PY{n}{y\PYZus{}axis}\PY{p}{,} \PY{n}{label}\PY{o}{=}\PY{l+s+s2}{\PYZdq{}}\PY{l+s+s2}{strat 3}\PY{l+s+s2}{\PYZdq{}}\PY{p}{)}
\PY{n}{plt}\PY{o}{.}\PY{n}{plot}\PY{p}{(}\PY{n}{x\PYZus{}axis}\PY{p}{,} \PY{n}{y\PYZus{}2}\PY{p}{,} \PY{n}{label}\PY{o}{=}\PY{l+s+s2}{\PYZdq{}}\PY{l+s+s2}{strat 2}\PY{l+s+s2}{\PYZdq{}}\PY{p}{)}
\PY{n}{plt}\PY{o}{.}\PY{n}{plot}\PY{p}{(}\PY{n}{x\PYZus{}axis}\PY{p}{,} \PY{n}{y\PYZus{}1}\PY{p}{,} \PY{n}{label} \PY{o}{=} \PY{l+s+s2}{\PYZdq{}}\PY{l+s+s2}{strat 1}\PY{l+s+s2}{\PYZdq{}}\PY{p}{)}
\PY{n}{plt}\PY{o}{.}\PY{n}{legend}\PY{p}{(}\PY{p}{)}
\end{Verbatim}
\end{tcolorbox}

    
    \begin{Verbatim}[commandchars=\\\{\}]
HBox(children=(FloatProgress(value=0.0, max=200.0), HTML(value='')))
    \end{Verbatim}

    
    \begin{Verbatim}[commandchars=\\\{\}]

    \end{Verbatim}

            \begin{tcolorbox}[breakable, size=fbox, boxrule=.5pt, pad at break*=1mm, opacityfill=0]
\prompt{Out}{outcolor}{ }{\boxspacing}
\begin{Verbatim}[commandchars=\\\{\}]
<matplotlib.legend.Legend at 0x7fee2e1a9630>
\end{Verbatim}
\end{tcolorbox}
        
    \begin{center}
    \adjustimage{max size={0.9\linewidth}{0.9\paperheight}}{output_52_3.png}
    \end{center}
    { \hspace*{\fill} \\}
    
    The three strategies perform around the same at
\texttt{q\ \textgreater{}=\ 0.7}. This is because our Strategy 3 uses
the fire map (the worst case future of the fire using
\texttt{q\ =\ 1.0}). At high \texttt{q} values such as \texttt{0.7} and
higher, the worst case fire map that we use in our Strategy 3 A*
heuristic is not much worse than the actual fire growth because the
actual \texttt{q} is very close to \texttt{1.0}. Because of this, we are
unable to find a viable path to the goal without visiting a node that
may be on fire. As a result, our Strategy 3 does about the same as the
other two strategies at these high \texttt{q} values. However, our
Strategy 3 performs better than or equal to the other two strategies at
\texttt{0.0\ \textless{}=\ q\ \textless{}=\ 0.7}. This is because we
assume the worst case fire spread. This means we are assuming a much
worse fire spread than the actual fire spread and as a result, the agent
chooses a much safer route than Strategies 1 and 2 at these \texttt{q}
values.

    \hypertarget{problem-7}{%
\subsection{Problem 7}\label{problem-7}}

    \hypertarget{if-you-had-unlimited-computational-resources-at-your-disposal-how-could-you-improve-on-strategy-3}{%
\subsubsection{If you had unlimited computational resources at your
disposal, how could you improve on Strategy
3?}\label{if-you-had-unlimited-computational-resources-at-your-disposal-how-could-you-improve-on-strategy-3}}

    Due to having limited computational resources, our current Strategy 3
only looks at the most optimal paths to the goal cell. We then pass
those optimal paths to our risk assessment algorithm and pick the best
path. However, if we had unlimited computation resources, our Strategy 3
would instead look at all possible paths to the goal cell. We could then
pass every single path to our risk assessment algorithm which would
greatly increase the chance that a viable path to the goal cell is found
without catching on fire. For example, if a fire is near a spot that an
optimal path may go through, our current Strategy 3 may not be able to
escape the fire because it leverages optimal paths. However, with
unlimited resources, we could look at all possible paths, therefore
avoiding the fire by discovering a less risky but non-optimal path.

    \hypertarget{problem-8}{%
\subsection{Problem 8}\label{problem-8}}

    \hypertarget{if-you-could-only-take-ten-seconds-between-moves-rather-than-doing-as-much-computation-as-you-like-how-would-that-change-your-strategy-describe-such-a-potential-strategy-4.}{%
\subsubsection{If you could only take ten seconds between moves (rather
than doing as much computation as you like), how would that change your
strategy? Describe such a potential Strategy
4.}\label{if-you-could-only-take-ten-seconds-between-moves-rather-than-doing-as-much-computation-as-you-like-how-would-that-change-your-strategy-describe-such-a-potential-strategy-4.}}

    Currently, every time we take a step, we recalculate all the optimal
paths using A* and recalculate the worst case fire map using the current
fire. This takes a lot time and may take more than 10 seconds between
steps. However, if we could only take ten seconds between moves rather
than doing as much computation as we like, we would just calculate the
next 5 steps and use that to decide which path to take, instead of
calculating the entire optimal path. Due to the fact that calculating
the entire fire map is extremely quick, we would not need to cut down on
how often we recalculate the fire map because we would still be able to
do this in less than 10 seconds between moves. As a result, our Strategy
4 would calculate just the next 5 steps using A* rather than calculating
the entire optimal path to the goal in between steps. This would greatly
cut down how much time we take between moves.


    % Add a bibliography block to the postdoc
    
    
    
\end{document}
